\subsection{Mobile Environment}
For mobile environment, we devise our system to work on a real android device. We compile the altered Chromium code for Android device. Certain steps that are involved in the push notification service work differently in a mobile environment. Firstly, while handling user interaction with notification (steps 7 \& 8), desktop environment requires the Chromium process to be running in order to display the push message sent by any website. However, In Android device, mobile operating system will generate the notification pop-up for user as a system notification and the application (Chromium) itself is not activated until someone clicks on the notification. In desktop environment, we used Puppeteer framework to automatically interact with the chromium browser to browser URLs. On the other hand, in Mobile, we use Android Debug Bridge (ADB) to start the chromium browser and enter URL into the browser. Since a service worker spawns as a different process than chrome, the requests (step 3) made by service worker script are not logged by the chromium browser itself and is not collected in mobile environment. Different components involved in mobile environment are explained below. 

\textbf{\indent{Enabling Notification Click: }}
In order to simulate user behavior to interact with the notifications, we developed an Android application that leverages the Android accessibility service. The accessibility service is aimed to help people with disabilities in using the device and apps in it. It is a long-run privileged system service that helps users process information from the screen and lets them interact with the content meaningfully in an easier way. An android developer can leverage the accessibility service API and develop apps that are made aware of certain events such as \texttt{TYPE\_VIEW\_FOCUSED} and \texttt{TYPE\_NOTIFICATION\_STATE\_CHANGED}. Further, it could also be used to initiate user actions such as click, touch and swipe. Our application that is pre-installed on the device has the permission of accessibility service and monitors every notification event fired from Chromium. Whenever a new notification pops up, our application will automatically swipe down the notification bar and click on the notification to complete the action. 

\textbf{\indent{Generating Event Logs: }}
The modified chromium code that logs events related to notification service along with the logs generated by the JSGraph\ref{jsgraph} version are the same for desktop and mobile browser. We built and compiled the same code for Android setup. However, the logs are generated in a different manner. The logs generated from every application in the device is collected by using the \texttt{logcat} command in ADB tool and piped into a destination file. \texttt{logcat} is a developer tool letting the developers print messages for debugging purpose. All the logs of chromium from different tabs, in addition to all the logs from other applications are piped into a single log file. We then later translate this file into desktop-version logs with only chromium logs. In this case, we can use one parser to analyze both desktop and mobile logs. Although the logs from different websites are mixed in one log file, we still can use post-analysis to successfully separate logs from each website.



